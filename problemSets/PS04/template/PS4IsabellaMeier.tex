\documentclass[12pt,letterpaper]{article}
\usepackage{graphicx,textcomp}
\usepackage{natbib}
\usepackage{setspace}
\usepackage{fullpage}
\usepackage{color}
\usepackage[reqno]{amsmath}
\usepackage{amsthm}
\usepackage{fancyvrb}
\usepackage{amssymb,enumerate}
\usepackage[all]{xy}
\usepackage{endnotes}
\usepackage{lscape}
\newtheorem{com}{Comment}
\usepackage{float}
\usepackage{hyperref}
\newtheorem{lem} {Lemma}
\newtheorem{prop}{Proposition}
\newtheorem{thm}{Theorem}
\newtheorem{defn}{Definition}
\newtheorem{cor}{Corollary}
\newtheorem{obs}{Observation}
\usepackage[compact]{titlesec}
\usepackage{dcolumn}
\usepackage{tikz}
\usetikzlibrary{arrows}
\usepackage{multirow}
\usepackage{xcolor}
\newcolumntype{.}{D{.}{.}{-1}}
\newcolumntype{d}[1]{D{.}{.}{#1}}
\definecolor{light-gray}{gray}{0.65}
\usepackage{url}
\usepackage{listings}
\usepackage{color}

\definecolor{codegreen}{rgb}{0,0.6,0}
\definecolor{codegray}{rgb}{0.5,0.5,0.5}
\definecolor{codepurple}{rgb}{0.58,0,0.82}
\definecolor{backcolour}{rgb}{0.95,0.95,0.92}

\lstdefinestyle{mystyle}{
	backgroundcolor=\color{backcolour},   
	commentstyle=\color{codegreen},
	keywordstyle=\color{magenta},
	numberstyle=\tiny\color{codegray},
	stringstyle=\color{codepurple},
	basicstyle=\footnotesize,
	breakatwhitespace=false,         
	breaklines=true,                 
	captionpos=b,                    
	keepspaces=true,                 
	numbers=left,                    
	numbersep=5pt,                  
	showspaces=false,                
	showstringspaces=false,
	showtabs=false,                  
	tabsize=2
}
\lstset{style=mystyle}
\newcommand{\Sref}[1]{Section~\ref{#1}}
\newtheorem{hyp}{Hypothesis}


\title{Problem Set 4}
\date{Due: December 3, 2023}
\author{Applied Stats/Quant Methods 1}


\begin{document}
	\maketitle
	\section*{Instructions}
	\begin{itemize}
		\item Please show your work! You may lose points by simply writing in the answer. If the problem requires you to execute commands in \texttt{R}, please include the code you used to get your answers. Please also include the \texttt{.R} file that contains your code. If you are not sure if work needs to be shown for a particular problem, please ask.
		\item Your homework should be submitted electronically on GitHub.
		\item This problem set is due before 23:59 on Sunday December 3, 2023. No late assignments will be accepted.
	\end{itemize}



	\vspace{.5cm}
\section*{Question 1: Economics}
\vspace{.25cm}
\noindent 	
In this question, use the \texttt{prestige} dataset in the \texttt{car} library. First, run the following commands:

\begin{verbatim}
install.packages(car)
library(car)
data(Prestige)
help(Prestige)
\end{verbatim} 


\noindent We would like to study whether individuals with higher levels of income have more prestigious jobs. Moreover, we would like to study whether professionals have more prestigious jobs than blue and white collar workers.

\newpage
\begin{enumerate}
	
	\item [(a)]
	Create a new variable \texttt{professional} by recoding the variable \texttt{type} so that professionals are coded as $1$, and blue and white collar workers are coded as $0$ (Hint: \texttt{ifelse}).
\begin{lstlisting}l
Prestige$Professional <- ifelse(Prestige$type == "prof", 1, 0)
Prestige <- subset(Prestige, select = -type)
head(Prestige)
tail(Prestige)	                                
\end{lstlisting}	                   
\begin{verbatim}
	                    education  income  women  prestige  census  Professional
gov.administrators      13.11  12351   11.16     68.8   1113            1
general.managers        12.26  25879    4.02     69.1   1130            1
accountants             12.77   9271   15.70     63.4   1171            1
purchasing.officers     11.42   8865    9.11     56.8   1175            1
chemists                14.62   8403   11.68     73.5   2111            1
physicists              15.64  11030    5.13     77.6   2113            1
                    education  income  women  prestige  census  Professional
train.engineers         8.49   8845     0.00     48.9   9131            0
bus.drivers             7.58   5562     9.47     35.9   9171            0
taxi.drivers            7.93   4224     3.59     25.1   9173            0
longshoremen            8.37   4753     0.00     26.1   9313            0
typesetters            10.00   6462    13.58     42.2   9511            0
bookbinders             8.55   3617    70.87     35.2   9517            0
\end{verbatim}
	
	
	\item [(b)]
	Run a linear model with \texttt{prestige} as an outcome and \texttt{income}, \texttt{professional}, and the interaction of the two as predictors (Note: this is a continuous $\times$ dummy interaction.)
\begin{lstlisting}
model <- lm(prestige ~ income + Professional + income*Professional, data = Prestige)
summary(model)
\end{lstlisting}
\begin{verbatim}
Residuals:   
           Min      1Q   Median      3Q     Max
	        -14.852  -5.332  -1.272   4.658  29.932 
   
Coefficients:                      
	                     Estimate   Std. Error t value Pr(>|t|)    
(Intercept)         21.1422589  2.8044261   7.539 2.93e-11 ***
income               0.0031709  0.0004993   6.351 7.55e-09 ***
Professional        37.7812800  4.2482744   8.893 4.14e-14 ***
income:Professional -0.0023257  0.0005675  -4.098 8.83e-05 ***

Residual standard error: 8.012 on 94 degrees of freedom
  (4 observations deleted due to missingness)
Multiple R-squared:  0.7872,	Adjusted R-squared:  0.7804 
F-statistic: 115.9 on 3 and 94 DF,  p-value: < 2.2e-16
\end{verbatim}
	\item [(c)]
	Write the prediction equation based on the result.
\begin{equation}
Y_i = b_0 + b_1*x_i +b_2*d_i + b_3*x_i*d_i
\end{equation}
\begin{equation}
prestige = 21.14 + 0.003*income+ 37.78*Professional + -0.002*income*Professional
\end{equation}	
	
\newpage
	\item [(d)]
	Interpret the coefficient for \texttt{income}.
\begin{verbatim}
	The coefficient for income of 0.003 means that a 1 unit increase in income
	is associated with a 0.003 increase in Prestige, holding other variables constant.
\end{verbatim}
	\item [(e)]
	Interpret the coefficient for \texttt{professional}.
\begin{verbatim}
	The coefficient for Professional suggests that the professional status 
	(1 instead of zero, professionals not blue or white collar) is asociated with a 
	37.78 increase in prestige holding other variables constant.
\end{verbatim}
	
	\newpage
	\item [(f)]
	What is the effect of a \$1,000 increase in income on prestige score for professional occupations? In other words, we are interested in the marginal effect of income when the variable \texttt{professional} takes the value of $1$. Calculate the change in $\hat{y}$ associated with a \$1,000 increase in income based on your answer for (c).
\begin{equation}
	prestige = 21.14 + 0.003*1 + 37.78*1 + (0.002*1*1)
\end{equation}
\begin{equation}
	prestige = 58.92
\end{equation}
\begin{equation}
	prestige = 21.14 + 0.003*1000 +37.78*1 + (-0.002*1000*1)
\end{equation}
\begin{equation}
	prestige = 61.92
\end{equation}
\begin{verbatim}
	For the sake of comparison, first I assumed that the base value of income is 1,
	because we want to know the effect of a $1000 increase. So with an assumed
	base income, prestige should be 58.92, considering the variable Professional
	to be 1, which is what we want to compare. An increase in $1000 is associated
	with an expected increase of 3 units in average prestige, holding
	Professional constant and (^y) according to the prediction equation in (c)
\end{verbatim}	
	
	\item [(g)]
	What is the effect of changing one's occupations from non-professional to professional when her income is \$6,000? We are interested in the marginal effect of professional jobs when the variable \texttt{income} takes the value of $6,000$. Calculate the change in $\hat{y}$ based on your answer for (c).
\begin{equation}
	prestige_np = 21.14 + 0.003*6000 +37.78*0 + (-0.002*6000*0)
\end{equation}
\begin{equation}
	prestige_np= 39.14
\end{equation}
\begin{equation}
  prestige_pro = 21.14 + 0.003*6000 + 37.78*1 + (-0.002*6000*1)
\end{equation}
\begin{equation}
prestige_pro = 64.92
\end{equation}
\begin{verbatim}
The marginal effect of changing someone's occupation from non_professional 
to professional is an expected increase 25.78 units in average prestige, when
income is held constant.
\end{verbatim}

	
\end{enumerate}

\newpage

\section*{Question 2: Political Science}
\vspace{.25cm}
\noindent 	Researchers are interested in learning the effect of all of those yard signs on voting preferences.\footnote{Donald P. Green, Jonathan	S. Krasno, Alexander Coppock, Benjamin D. Farrer,	Brandon Lenoir, Joshua N. Zingher. 2016. ``The effects of lawn signs on vote outcomes: Results from four randomized field experiments.'' Electoral Studies 41: 143-150. } Working with a campaign in Fairfax County, Virginia, 131 precincts were randomly divided into a treatment and control group. In 30 precincts, signs were posted around the precinct that read, ``For Sale: Terry McAuliffe. Don't Sellout Virgina on November 5.'' \\

Below is the result of a regression with two variables and a constant.  The dependent variable is the proportion of the vote that went to McAuliff's opponent Ken Cuccinelli. The first variable indicates whether a precinct was randomly assigned to have the sign against McAuliffe posted. The second variable indicates
a precinct that was adjacent to a precinct in the treatment group (since people in those precincts might be exposed to the signs).  \\

\vspace{.5cm}
\begin{table}[!htbp]
	\centering 
	\textbf{Impact of lawn signs on vote share}\\
	\begin{tabular}{@{\extracolsep{5pt}}lccc} 
		\\[-1.8ex] 
		\hline \\[-1.8ex]
		Precinct assigned lawn signs  (n=30)  & 0.042\\
		& (0.016) \\
		Precinct adjacent to lawn signs (n=76) & 0.042 \\
		&  (0.013) \\
		Constant  & 0.302\\
		& (0.011)
		\\
		\hline \\
	\end{tabular}\\
	\footnotesize{\textit{Notes:} $R^2$=0.094, N=131}
\end{table}

\vspace{.5cm}
\begin{enumerate}
	\item [(a)] Use the results from a linear regression to determine whether having these yard signs in a precinct affects vote share (e.g., conduct a hypothesis test with $\alpha = .05$).
\begin{verbatim}
	With the coefficient as 0.042 and the standard error as 0.016 we can do two tailed
	t-test where:
	t = 0.042/0.016 = 2.62
	With the degree of freedom as df = N - k - 1 = 128
	Using a t-test table the critical value is 1.98. The t-test value is greater than
	the critical value. 
	The results of the linear regression suggest that the presence of the yard 
	signs affects Ken Cuccinelli's vote share and the result of the t-test suggests 
	that the effect is statistically significant and we can reject the null hyphotesis.
	The null hyphothesis would be that the yard signs have no effect on Ken
	Cuccinelli's vote share. The t-test is two tailed because the question states that
	we want to learn "the effect" of the yard signs over vote share.
\end{verbatim}
	\newpage		
	\item [(b)]  Use the results to determine whether being
	next to precincts with these yard signs affects vote
	share (e.g., conduct a hypothesis test with $\alpha = .05$).
\begin{verbatim}
	With the coefficient as 0.042 and the standard error as 0.013. we can do a two
	tailed t-test where:
	t = 0.042/0.013 = 3.23
	With the degree of freedom as df = 128
	Using a t-test table the critical value is 1.98. The t-test value is greater than
	the critical value. 
	The results of the linear regression suggest that the presence of adjacent yard 
	signs affects Ken Cuccinelli's vote share and the result of the t-test suggests 
	that the effect is statistically significant and we can reject the null hyphotesis.
\end{verbatim}
	\item [(c)] Interpret the coefficient for the constant term substantively.
\begin{verbatim}
	The coefficient fo the constant term means that on average, the baseline of
	Ken C."s vote share is 0.302 when two variables are at their reference levels.
\end{verbatim}	
	
	\item [(d)] Evaluate the model fit for this regression.  What does this	tell us about the importance of yard signs versus other factors that are not modeled?
\begin{verbatim}
	The Rsquared value 0.094 indicates that 9.4% of the variation in Ken C's 
	vote share is explained by the yard signs. The other 90.6% of the variation is not
	explained by the model. We don't have enough information to infer the importance
	of yard signs against other models. 
\end{verbatim}
	
\end{enumerate}  


\end{document}
